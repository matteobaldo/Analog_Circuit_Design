\chapter{Filters}

\begin{equation}
2\pi f=\omega
\end{equation}

\section{Parameters}

\centering
\includegraphics[width=0.5\textwidth]{mask.png}\\
\raggedright

Selectivity index
\begin{equation}
k=\frac{\omega_{BP}}{\omega_{SB}}
\end{equation}

Attenuation index
\begin{equation}
\varepsilon_{BP}=\sqrt{10^{A_{BP}/10}-1} \ \ \ \ \ \ \ \varepsilon_{BP}=\sqrt{10^{A_{SB}/10}-1}
\end{equation}

Discriminator factor 
\begin{equation}
k_{\varepsilon}=\frac{\varepsilon_{BP}}{\varepsilon_{SB}}<1
\end{equation}
Higher the attenuation in stop-band lower $k_{\varepsilon}$ is.

%------------------------------------------------------------------------%
\section{Butterworth Filters}
%------------------------------------------------------------------------%
Maximum flatness in passband.\\

General formula for the module 
\begin{equation}
|H(j\omega)|=\frac{1}{\sqrt{1+(\frac{\omega}{\Omega_0})^{2n}}}
\end{equation}
where n is the filter order.\\

To mach the in-band attenuation spec. we have to respect
\begin{equation}
\frac{\Omega_{BP}}{\Omega_0}\le \varepsilon_{BP}^{1/n}
\end{equation}

To mach the stop-band attenuation spec. we have to respect
\begin{equation}
\frac{\Omega_{SB}}{\Omega_0}\ge \varepsilon_{SB}^{1/n}
\end{equation}

So we get an interval of $\Omega_0$ and a formula for the filter order
\begin{equation}
\frac{\Omega_{BP}}{\varepsilon_{BP}^{1/n}} \ge \Omega_0 \le \frac{\Omega_{SB}}{\varepsilon_{SB}^{1/n}} \ \ \ \ \ \ \ \ \ \ \ \ \ \ n\le \frac{\ln(k_{\varepsilon})}{\ln(k)}
\end{equation}

Poles in the Gauss plane are disposed as in figure


\centering
\includegraphics[width=0.35\textwidth]{Gaussbutt.png}\\
\raggedright

Table of Butterworth polynomials

\centering
\includegraphics[width=0.5\textwidth]{butt.png}\\
\raggedright


%------------------------------------------------------------------------%
\section{Bessel Filters}
%------------------------------------------------------------------------%

Maximum phase linearity.\\
General formula for the module 
\begin{equation}
|H(s)|=J_n(0)/J_n(s)
\end{equation}
where n is the filter order.

Table of poles for a n order Bessel 

\centering
\includegraphics[width=0.35\textwidth]{bessel.png}\\
\raggedright

%------------------------------------------------------------------------%
\section{Chebyshev Type-I}
%------------------------------------------------------------------------%
Low number of poles good selectivity but bad phase.\\
General formula is
\begin{equation}
H(s)=\frac{K_n}{D_n(s)}
\end{equation}
where $D_n(s)$ are the Cheby. polynomials and $K_n$ is a coefficient to have $|H(0)|=1$ for filters of odd order and $|H(0)|=1/\sqrt{1+\varepsilon_{BP}^2}$ for even filters order (count the ripple).\\
\vspace{2mm}
Order of the filter form the specs can be derived as
\begin{equation}
n\ge \frac{arCosh(k_{\varepsilon}^{-1})}{arCosh(k^{-1})}
\end{equation} 

Poles derived from the following formulas to mach exactly the BP attenuation

\centering
\includegraphics[width=0.15\textwidth]{gamma.png}\\
\raggedright


\centering
\includegraphics[width=0.45\textwidth]{s.png}\\
\raggedright

where we remember that 
\begin{equation}
\omega=\sqrt{r^2+i^2} \ \ \ \ \ \ \ Q=\frac{|omega|}{|2r|}
\end{equation}

Where k ranges form 1,2,...2n and only poles with negative real part have to be considered.\\

To mach SB attenuation exactly we have to use in $\Gamma$
\begin{equation}
\varepsilon_{BP}'=\frac{\varepsilon_{SB}}{Ch(n*arCh(1/k))}
\end{equation}


\centering
\includegraphics[width=0.5\textwidth]{cheby.png}\\
\raggedright

Poles in Gauss plane placed in an ellipse.\\





%------------------------------------------------------------------------%
%------------------------------------------------------------------------%
\section{Sallen-Key Cell}

\subsection{With K$>$1}
\centering
\includegraphics[width=0.35\textwidth]{skg.png}\\
\raggedright

With all R equal ,all capacitors equal C and K the gain of the amplifier 
\begin{equation}
\omega_0=\frac{1}{RC}\ \ \ \ \ \ \ \ \ \  \ \ \ \ \ \ \ \ Q=\frac{RC}{RC(k-1)+2RC}
\end{equation}
Form the Q equation we get the dependaces with k 
\begin{equation}
k=3-\frac{1}{Q} \ \ \ \ \ \ Q=\frac{1}{(3-k)}
\end{equation}
A variation of $\Delta k/k=\%$ is mapped in a Q variation of 

\begin{equation}
\Delta Q \simeq \Delta k \cdot Q^2
\end{equation}
The most significant change in the transfer function due to a change of Q is at the cut off frequency $\Omega_{bp}$; there we can define the proportion
\begin{equation}
10\% \ \ of \ \ \frac{\Delta Q}{Q} \propto 1dB
\end{equation}
Highes the Q bigger the variation for a given percentage bigger attenuation.\\


\subsection{With K=1}
\centering
\includegraphics[width=0.35\textwidth]{skb.png}\\
\raggedright

With all R equal and K=1 the capacitor has to be different to add a degree of freedom to the system so
\begin{equation}
\omega_0=\frac{1}{\sqrt{n}RC}\ \ \ \ \ \ \ \ \ \ \ \ \ \ \ \ \ Q=\frac{\sqrt{Q}}{2}
\end{equation}

\section{Non idealities}
When we get a non ideal op-amp with finite GBWP and DC gain and olso others poles or positive zeros this imperfections leads us to a change of Q like in the relations 
\begin{equation}
\frac{\Delta Q}{Q}\simeq 2Q\left(\frac{f_0}{GBWP}+\frac{f_0}{f_z}+\frac{f_z}{f_p}-\frac{1}{A_0}\right)
\end{equation}
